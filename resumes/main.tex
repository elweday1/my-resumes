\documentclass[10pt, letterpaper]{article}

% Packages:
\usepackage[
    ignoreheadfoot, % set margins without considering header and footer
    top=2 cm, % seperation between body and page edge from the top
    bottom=2 cm, % seperation between body and page edge from the bottom
    left=2 cm, % seperation between body and page edge from the left
    right=2 cm, % seperation between body and page edge from the right
    footskip=1.0 cm, % seperation between body and footer
    % showframe % for debugging 
]{geometry} % for adjusting page geometry
\usepackage{titlesec} % for customizing section titles
\usepackage{tabularx} % for making tables with fixed width columns
\usepackage{array} % tabularx requires this
\usepackage[dvipsnames]{xcolor} % for coloring text
\definecolor{primaryColor}{RGB}{0, 0, 0} % define primary color
\usepackage{enumitem} % for customizing lists
\usepackage{fontawesome5} % for using icons
\usepackage{amsmath} % for math
\usepackage[
    pdftitle={Mohammed Nasser's Resume},
    pdfauthor={Mohammed Nasser},
    colorlinks=true,
    urlcolor=primaryColor
]{hyperref} % for links, metadata and bookmarks
\usepackage[pscoord]{eso-pic} % for floating text on the page
\usepackage{calc} % for calculating lengths
\usepackage{bookmark} % for bookmarks
\usepackage{lastpage} % for getting the total number of pages
\usepackage{changepage} % for one column entries (adjustwidth environment)
\usepackage{paracol} % for two and three column entries
\usepackage{ifthen} % for conditional statements
\usepackage{needspace} % for avoiding page brake right after the section title
\usepackage{iftex} % check if engine is pdflatex, xetex or luatex

% Ensure that generate pdf is machine readable/ATS parsable:
\ifPDFTeX
    \input{glyphtounicode}
    \pdfgentounicode=1
    \usepackage[T1]{fontenc}
    \usepackage[utf8]{inputenc}
    \usepackage{lmodern}
\fi

\usepackage{charter}

% Some settings:
\raggedright
\AtBeginEnvironment{adjustwidth}{\partopsep0pt} % remove space before adjustwidth environment
\pagestyle{empty} % no header or footer
\setcounter{secnumdepth}{0} % no section numbering
\setlength{\parindent}{0pt} % no indentation
\setlength{\topskip}{0pt} % no top skip
\setlength{\columnsep}{0.15cm} % set column seperation
\pagenumbering{gobble} % no page numbering

\titleformat{\section}{\needspace{4\baselineskip}\bfseries\large}{}{0pt}{}[\vspace{1pt}\titlerule]

\titlespacing{\section}{
    % left space:
    -1pt
}{
    % top space:
    0.3 cm
}{
    % bottom space:
    0.2 cm
} % section title spacing

\renewcommand\labelitemi{$\vcenter{\hbox{\small$\bullet$}}$} % custom bullet points
\newenvironment{highlights}{
    \begin{itemize}[
        topsep=0.10 cm,
        parsep=0.10 cm,
        partopsep=0pt,
        itemsep=0pt,
        leftmargin=0 cm + 10pt
    ]
}{
    \end{itemize}
} % new environment for highlights


\newenvironment{highlightsforbulletentries}{
    \begin{itemize}[
        topsep=0.10 cm,
        parsep=0.10 cm,
        partopsep=0pt,
        itemsep=0pt,
        leftmargin=10pt
    ]
}{
    \end{itemize}
} % new environment for highlights for bullet entries

\newenvironment{onecolentry}{
    \begin{adjustwidth}{
        0 cm + 0.00001 cm
    }{
        0 cm + 0.00001 cm
    }
}{
    \end{adjustwidth}
} % new environment for one column entries

\newenvironment{twocolentry}[2][]{
    \onecolentry
    \def\secondColumn{#2}
    \setcolumnwidth{\fill, 4.5 cm}
    \begin{paracol}{2}
}{
    \switchcolumn \raggedleft \secondColumn
    \end{paracol}
    \endonecolentry
} % new environment for two column entries

\newenvironment{threecolentry}[3][]{
    \onecolentry
    \def\thirdColumn{#3}
    \setcolumnwidth{, \fill, 4.5 cm}
    \begin{paracol}{3}
    {\raggedright #2} \switchcolumn
}{
    \switchcolumn \raggedleft \thirdColumn
    \end{paracol}
    \endonecolentry
} % new environment for three column entries

\newenvironment{header}{
    \setlength{\topsep}{0pt}\par\kern\topsep\centering\linespread{1.5}
}{
    \par\kern\topsep
} % new environment for the header

\newcommand{\placelastupdatedtext}{% \placetextbox{<horizontal pos>}{<vertical pos>}{<stuff>}
  \AddToShipoutPictureFG*{% Add <stuff> to current page foreground
    \put(
        \LenToUnit{\paperwidth-2 cm-0 cm+0.05cm},
        \LenToUnit{\paperheight-1.0 cm}
    ){\vtop{{\null}\makebox[0pt][c]{
        \small\color{gray}\textit{Last updated in September 2024}\hspace{\widthof{Last updated in September 2024}}
    }}}%
  }%
}%

    \newcommand{\AND}{\unskip
        \cleaders\copy\ANDbox\hskip\wd\ANDbox
        \ignorespaces
    }
    \newsavebox\ANDbox
    \sbox\ANDbox{$|$}


% save the original href command in a new command:
\let\hrefWithoutArrow\href

% new command for external links:


\begin{document}
\begin{header}
    \fontsize{25 pt}{25 pt}\selectfont Mohammed Nasser Hilal

    \normalsize
    \kern 5.0 pt%
    \mbox{\hrefWithoutArrow{mailto:mohammednh284@gmail.com}{mohammednh284@gmail.com}}%
    \kern 5.0 pt%
    \AND%
    \kern 5.0 pt%
    \mbox{Cairo, EG}%
    \kern 5.0 pt%
    \AND%
    \kern 5.0 pt%
    \mbox{\hrefWithoutArrow{https://linkedin.com/in/elweday}{linkedin.com/in/elweday}}%
    \kern 5.0 pt%
    \AND%
    \kern 5.0 pt%
    \mbox{\hrefWithoutArrow{https://github.com/elweday1}{github.com/elweday1}}%


\end{header}

\section{Education}




\begin{twocolentry}{
        Oct 2021 – Aug 2026
    }
    \textbf{Helwan University}, B.S. in Computer Engineering\end{twocolentry}

\vspace{0.10 cm}
\begin{onecolentry}
    \begin{highlights}
        \item \textbf{Coursework:} Operating Systems, Object Oriented Programming, Data Structures and Algorithms, Databases, Computer Architecture, Microprocessors Microcontrollers
    \end{highlights}
\end{onecolentry}

\section{Experience}

\begin{twocolentry}{
        Aug 2024 – Present
    }
    \textbf{Software Engineer Long-Term Intern}, Siemens -- Cairo, EG (Hybrid)\end{twocolentry}

\vspace{0.10 cm}
\begin{onecolentry}
    \begin{highlights}
        \item Contributed to the development of a next-generation IDE for hardware description languages by migrating from a legacy C++ QT client while abstracting core logic into reusable services to be used in a web-based client.

        \item Engineered a robust treeview component for parsing, selecting, and managing design templates from the file system, streamlining the development process for hardware designers.
        \item Developed gRPC services and clients to seamlessly integrate C++ backend with TypeScript frontend, enabling real-time communication and facilitating the separation of UI from core functionality in a complex software tool.

    \end{highlights}
\end{onecolentry}


\vspace{0.4 cm}

\begin{twocolentry}{
        Nov 2023 – Aug 2024
    }
    \textbf{Software Engineer}, Studyo -- London, UK (Remote)\end{twocolentry}

\vspace{0.10 cm}
\begin{onecolentry}
    \begin{highlights}
        \item Orchestrated the development of backend video processing containerized microservices on Google Cloud Platform using Go and Python to serve as a backend for a mobile applications and internal services.
        \item Integrated CI/CD pipelines with GitHub Actions and Docker, cutting deployment times by 50\% and significantly reducing integration errors.
        \item Developed and documented CLI tools and modular solutions to automate repetitive development tasks, saving the team 10 hours per week on average.
        \item Fine-tuned the custom LLMs for the company's e-learning platform, and interfaced with the OpenAI, GCP APIs to improve the preformance of the models.
    \end{highlights}
\end{onecolentry}

\section{Projects}
\textbf{\href{https://github.com/elweday1/scribble}{Scribble: Collaborative Drawing Game}}
\vspace{0.10 cm}
\begin{onecolentry}
    \begin{highlights}
        \item Engineered a real-time multiplayer drawing and guessing game using Next.js, React, and WebSockets.
        \item Leveraged YJS for synchronizing the drawing state between peers providing a conflict-free real-time collaboration, enabling seamless drawing experiences across devices.
        \item Implemented a custom state machine to manage game logic, ensuring smooth and efficient gameplay.

    \end{highlights}
\end{onecolentry}


\vspace{0.2 cm}

\textbf{\href{https://github.com/elweday1/CollegeSite}{College Management System}}

\vspace{0.10 cm}
\begin{onecolentry}
    \begin{highlights}
        \item Architected a comprehensive academic portal using SvelteKit and PostgreSQL, providing a robust platform for educators and students.
        \item Designed an intuitive UI for course registration, grade tracking, and timetable management.
        \item Integrated data visualization tools, providing insightful analytics on student performance and course popularity.
        \item Developed and maintained a comprehensive backend API, ensuring seamless integration with the frontend.
    \end{highlights}
\end{onecolentry}


\section{Skills}


\textbf{Languages}{: JavaScript, TypeScript, Python, Go, C, C++, Java} \\ \vspace{0.2 cm}
\textbf{Web}{:  HTML, CSS, React, Node.js, Express, Next.js, Django, Flask.} \\ \vspace{0.2 cm}
\textbf{Cloud \& DevOps}{: Docker, CI/CD, GCP(iAM, Vertex, Run, Functions, Compute, Storage, Firestore), AWS, Github Actions} \\ \vspace{0.2 cm}
\textbf{Data \& APIs}{: SQL, MongoDB, NoSQL, ORMs, DBMS, REST, gRPC, GraphQL, Websockets, Even-Driven.} \\ \vspace{0.2 cm}
\textbf{Other}{: Git, Github, Linux, Docker, Design Patterns, Web Scraping, Unit Testing, Scripting, Microservices} \\ \vspace{0.2 cm}

\end{document}
